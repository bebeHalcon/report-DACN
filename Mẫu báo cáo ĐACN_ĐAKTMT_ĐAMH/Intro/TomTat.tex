\section*{Tóm tắt}
\thispagestyle{empty}

\hspace*{0.5cm} SQL đang trở thành một trong những ngôn ngữ được sử dụng phổ biến nhất hiện nay trong lĩnh vực kỹ thuật dữ liệu. Tuy nhiên, việc học một ngôn ngữ declarative như SQL có thể gây nhàm chán, do người học chỉ xoay quanh việc học lý thuyết về các câu truy vấn, các biểu thức trong câu truy vấn, cách này có thể giúp người học nhớ được cấu trúc của cú pháp câu truy vấn, hoặc syntax của biểu thức. Một bộ phận người học cũng muốn vừa học ngôn ngữ SQL kết hợp với việc luyện tập tư duy Logic trong việc giải quyết các bài toán từ dễ đến khó thông qua SQL, muốn tìm một công cụ luyện tập lý thú, vừa chơi vừa học. Với mong muốn mang dến một trải nghiệm vừa học vừa chơi lý thú, Nhóm đã phát triển trò chơi MeowSQL Knight, một trò chơi trò chơi chiến đấu theo lượt mới lạ. Người phải chiến đấu với các quái vật bằng các câu truy vấn của ngôn ngữ SQL (Structured Query Language) và đánh bại chúng. \\
\hspace*{0.5cm} Việc xây dựng cơ chế chơi và phát triển các giải phảp tích hợp việc thực thi câu truy vấn SQL ngay trong game là một thách thức lớn đồng thời cũng là cơ hội dành cho nhóm. Việc thiết kế cơ chế game đòi hỏi sự sáng tạo không ngừng nghỉ, khi phải kết hợp các lối chơi sẵn có trên thị trường, thêm thắt các điểm mới và điều chỉnh chúng sao cho thật hợp lý đã là một thách thức không nhỏ với nhóm. Với việc phát triển cốt lõi chính của trò chơi, nhóm đã sử dụng Unity Engine cùng với ngôn ngữ C\# làm nền tảng để xây dựng những tính năng chính của trò chơi.\\
\hspace*{0.5cm} Mặt khác, việc xây dựng màn chơi cũng phải là một điểm đáng lưu tâm, việc xây dựng màn chơi có một số chuẩn mực nhất định, mục đích là lôi cuốn người chơi vào game và khiến họ gắn bó lâu dài với game. Việc thiết kế màn chơi không chỉ là khoa học, mà nó là nghệ thuật trong lĩnh vực phát triển game.\\
\hspace*{1cm} Cốt lõi của game chính là SQL và Database, việc tích hợp việc thực thi câu truy vấn SQL ngay trong game đòi game phải chạy ổn định, nhất quán mà vẫn đáp ứng yêu cầu câu truy vấn từ người chơi. Giải pháp chạy SQL Server riêng nhìn chung có ưu thế về mặt tính năng, do nó có thể là MySQL hoặc PostgreSQL nên có thể cung cấp rất nhiều tính năng cho game như trong việc phân quyền. Song, do là một process riêng biệt nên rất dễ bị can thiệp từ bên ngoài, làm game trở nên kém ổn định, độ reliable thấp. Hơn nữa, đây là trò chơi đơn nên việc kết nối tới server với mỗi tương tác là thực sự không cần thiết. Vì vậy nhóm lựa chọn giải pháp thứ hai, đó là sử dụng SQLite, một hệ cơ sở dữ liệu có thể nhúng vào ứng dụng. Tuy tính năng có thể không đầy nhưng đối với nhóm lựa chọn này là chấp nhận được, nhóm cũng sẽ thiết lập các quy tắc quản lý từ tầng Manager của Game và xử lý các hành vi của người dùng. Những module này được nối với nhau nhờ Dependency Injection và liên kết bằng Observer Pattern.\\
\hspace*{1cm} Có thể nói đây là một dự án với nhiều tham vọng, và quy mô không thể gọi là nhỏ. Nhóm đã phải đối mặt với một số thách thức về chủ quan và khách quan, về công nghệ và con người. Dù chỉ gồm hai thành viên, nhóm đã nỗ lực tối đa để triển khai càng nhiều tính năng càng tốt trong thời gian hạn chế. Với khối lượng công việc lớn và yêu cầu kỹ thuật phức tạp, nhóm đã ưu tiên hoàn thiện phần lõi của hệ thống – những chức năng chủ đạo mang tính quyết định đến hiệu quả vận hành. Tuy nhiên, do nguồn lực hạn chế, một số tính năng phụ trợ và phần mở rộng vẫn chưa thể hoàn thiện như kỳ vọng. Nếu hoàn thiện, MeowSQL Knight sẽ là một trò chơi đầy sáng tạo, giúp người chơi vừa học SQL vừa chơi với đồ họa thân thiện và lối chơi đổi mới.\\

\clearpage
\pagenumbering{arabic}
