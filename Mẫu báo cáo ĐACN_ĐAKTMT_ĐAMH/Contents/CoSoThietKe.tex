\section{Cơ sở thiết kế trò chơi}
\subsection{Tổng quan trò chơi}
\hspace*{1cm} \textbf{MeowSQL Knight} là một trò chơi nhập vai Phiêu lưu chiến đấu theo lượt. Thay vì sử dụng các thao tác thông thường bằng các nút, người chơi tương tác bằng các nhập câu truy vấn và thực thi chúng, thu thập dữ kiện để chiếm lấy ưu thế trong chiến đấu, đánh bại quái vật và hoàn thành màn chơi.\\
\begin{figure}[H]
	\centering
	\includegraphics[width=\textwidth]{Images/Overall.png}
	\vspace{0.5cm}
	\caption{Tổng quan một màn chơi trong MeowSQL Knight}
\end{figure}
\hspace*{1cm} Người chơi sẽ sử dụng các câu truy vấn để khai thác schema cho sẵn. Sử dụng \textbf{select} để hiện nội dung các record lên màn hình, thu thập thông tin từ những dữ liệu đó. Ngoài ra cũng có thể \textbf{insert} và \textbf{delete} những record trên bảng và xem phản ứng của trò chơi sẽ như thế nào. Người chơi sẽ tấn công quái vật bằng các vũ khí mà người chơi mang vào trận chiến, tuy nhiên không thể tấn công quái vật một cách đơn thuần, người chơi sẽ phải tấn công vào các điểm yếu của quái vật nằm trên các bộ phận khác nhau. Để tương tác với game (tấn công quái vật, dùng vật phẩm), người chơi sẽ insert các tham số vào các bảng có chức năng đặc biệt, sau khi chèn xong thì hành động sẽ được thực thi.\\
\hspace*{1cm} Trong Schema này, các thực thể như người chơi, quái vật, bộ phận chí mạng của quái vật,... đều sử dụng chung ID có cấu trúc như nhau. Để tiêu diệt quái vật, người chơi phải nhắm đến các bộ phận chí mạng này, cũng như xác định được ID của chúng để có thể tấn công chính xác, bằng không các đòn đánh của người chơi sẽ bị trượt.\\
\hspace*{1cm} Người chơi phải tìm cách khai thác schema một cách hiệu quả trong chiến đấu. Đi kèm với việc lựa chọn trang bị hợp lý và tận dụng lợi thế môi trường, giành chiến thắng, hoàn thành màn chơi và đi sâu hơn để tìm ra bí ẩn của trò chơi.


\subsection{Sơ lược cốt truyện}
\hspace*{1cm} Trong một thế giới game RPG có chủ đề là động vật bình thường, bạn là một kỵ sĩ mèo đang đi rừng để tìm nguyên liệu, mọi thứ diễn ra một cách bình thường. \\
\hspace*{1cm} Đột nhiên, trò chơi bỗng hoạt động không đúng so với trước kia, các quái vật trở nên hung hãn hơn, nguy hiểm hơn, lẽ ra chỉ cần đánh bằng các phương pháp thông thường đã có thể diệt sạch chúng, nhưng kì lạ thay, các phương pháp thông thường không còn có hiệu quả với chúng nữa. Đám quái vật vùng lên làm loạn cả thế giới game, khiến thế giới game bị lỗi nghiêm trọng, và nếu cứ tiếp tục như vậy, trò chơi sẽ không thể chơi được nữa.\\
\hspace*{1cm} Chính lúc này, nhân vật chính của chúng ta bị một con slime bao vây, lẽ ra 1 nhát từ kiếm của cậu có thể hạ gục con Slime, tuy nhiên, trong thế giới game bị rối loạn như thế này là không thể. Cậu cứ đánh trong vô vọng, trong khi con quái vật giận dữ tiến gần định nuốt chửng cậu. Đột nhiên, một tia sáng xuất hiện giết chết con quái vật đó. Là Nhà phát triển (Dev) của trò chơi này. Anh ấy đã kiểm tra hệ thống thế giới game và phát hiện ra bạn là một trong những object hiếm hoi còn sống trong thế giới game này. Để hỗ trợ bạn, Dev cấp cho bạn 1 năng lực lớn: bạn có thể xem schema của game và thực hiện các câu truy vấn SQL để khai thác schema và chiến đấu. Vì chỉ có sử dụng SQL mới có thể đánh bại các quái vật SQL. \\
\hspace*{1cm} Dev muốn đặt niềm tin vào bạn vì Dev không thể khởi dộng lại Project game này, nó là một project chạy trên server nhưng cậu đã mất quyền kiểm soát cả server và project. Dev muốn bạn đi sâu vào bên trong lõi của game và tìm ra lý do khiến game trở nên như vậy. Bạn là hy vọng của thế giới game này, và là hy vọng của cả Dev. Bạn bắt đầu đi vào sâu trong thế giới, mang sứ mệnh vô cùng cao cả, không chỉ để cứu thế giới này.

\subsection{So Sánh các sản phẩm tương tự trên thị trường}
\subsubsection{Game 1}
\subsubsection{Game 2}
\subsubsection{Điểm khác biệt của ...}

\subsection{Luật chơi}
\hspace*{1cm} Mỗi màn chơi, người chơi sẽ được đưa đến một bản đồ, gồm các phòng liên thông với nhau. Nhiệm vụ của người chơi là sử dụng các câu truy vấn để tương tác nhân vật chính, hoàn thành yêu cầu được đưa ra trong mỗi căn phòng, đến điểm cuối của bản đồ và hoàn thành màn chơi.

\subsubsection{Bản đồ trong màn chơi}
\hspace*{1cm}Một màn chơi sẽ có dạng mê cung gồm nhiều căn phòng. Các căn phòng này có thể liên thông với nhau tạo nên một mê cung phức tạp cho người chơi khám phá. Người chơi khi bắt đầu màn chơi sẽ xuất phát ở căn phòng lối vào và kết thúc ở căn phòng lối ra. Chỉ có một lối vào và có thể có một hoặc nhiều lối ra. Ở lối ra thường sẽ có một con Miniboss hoặc Boss để người chơi đối đầu. Mặc định là các phòng bị bóng tối che khuất và chỉ có thể thấy các căn phòng khác khi đã clear được căn phòng hiện tại.\\

\begin{figure}[H]
	\centering
	\includegraphics[width=10cm]{Images/MapStructure.png}
	\vspace{0.5cm}
	\caption{Cấu trúc một bản đồ chơi}
\end{figure}

\hspace*{1cm}Trong một căn phòng, có thể xuất hiện quái vật, rương kho báu hoặc không có gì, đi cùng với một môi trường nhất định. Người chơi có thể chinh phục 1 căn phòng bằng các cách như tiêu diệt quái vật, mở rương kho báu hoặc đi vào 1 căn phòng trống. Sau khi chinh phục một căn phòng thì game sẽ mở khóa các căn phòng lân cận (không cần các phòng phải lối đi từ căn phòng hiện tại), người chơi quyết định đi căn phòng nào tiếp theo. Lưu ý rằng người chơi không được quay đầu (quay trở lại căn phòng đã đi trước đó). Các căn phòng có thể chứa các route đến các màn chơi khác trên bản đồ. Căn phòng cuối cùng, chính là lối ra, trong đó sẽ có một con boss hoặc min boss. Nhiệm vụ của người chơi là đánh bại boss đó để có thể thoát ra ngoài.\\
\subsubsection{Chiến đấu}
\hspace*{1cm}Gameplay chính của phần chiến đấu hoạt động theo dạng đánh theo lượt luân phiên giữa người chơi và quái vật.

\begin{itemize}
	\item Khi đến lượt của mình, người chơi được thực thi một câu lệnh hợp lệ của SQL. Câu lệnh phải chạy được thì mới tính là một câu lệnh hợp lệ. Sử dụng các các câu lệnh của SQL là chưa đủ, để tấn công quái vật, người chơi phải sử dụng câu lệnh gọi hàm được game cho trước (sẽ được nêu ở Game Mechanics). Mặc dù có cơ chế đánh theo lượt luân phiên, vẫn có khả năng người chơi sau khi xong một lượt có thể chơi một lượt nữa (đây gọi là sự nhanh nhẹn, nó sẽ là một chỉ số có ảnh hưởng đến xác suất đánh thêm lượt nữa của người chơi và quái vật). Để công bằng, cho dù người chơi có gia tăng chỉ số nhanh nhẹn cao bao nhiêu, xác suất tối đa để thêm lượt là 40\%. Điều này được tính cho cả quái vật. Tuy nhiên, mỗi thực thể chỉ có thể tấn công tối đa 2-3 lượt
	\item Đến lượt của quái, quái sẽ tấn công người chơi theo 2 cách
	\begin{itemize}
		\item Quái vật tạo ra chướng ngại hoặc thử thách để người chơi phải tránh né (tương tự với game Undertale). Người chơi có thể trúng nhiều hơn 1 đòn đánh của quái nhưng hiệu ứng sát thương có thể chỉ ảnh hưởng 1 lần và không cộng dồn trong lần tấn công của quái.
		
		\begin{figure}[H]
			\centering
			\includegraphics[width=10cm]{Images/MonsterAttack1.png}
			\vspace{0.5cm}
			\caption{Thử thách quái vật tạo ra}
		\end{figure}
		
		\item Quái sẽ tấn công người chơi 1 lần duy nhất trong lượt, người chơi chắc chắn trúng đòn (sát thương có thể ít hay nhiều, thậm chí không đáng kể, không ảnh hưởng health). Đòn tấn công này có thể mang một số hiệu ứng khác nhau (sẽ nêu rõ trong Game Mechanics). Cách tấn công này tương tự Bookworm Adventure.
		
		\begin{figure}[H]
			\centering
			\includegraphics[width=10cm]{Images/MonsterAttack2.png}
			\vspace{0.5cm}
			\caption{Quái vật tấn công kèm hiệu ứng}
		\end{figure}
		
	\end{itemize}
\end{itemize}
\subsubsection{Win condition}
Người chơi sẽ thắng khi đánh bại được quát vật, tức là khiến cho hp của nó về 0.
\subsubsection{Lose condition}
Người chơi sẽ thua khi bị tiêu diệt bởi quái vật, để hp của mình chạm đến 0.
\subsection{Các đối tượng chính trong màn chơi}